\documentclass[a4paper]{article}

\usepackage{hyperref}
\usepackage[bottom=5em]{geometry}
\usepackage{enumitem}

% \usepackage[T1]{fontenc}

\def\name{Steffen Reindl}

% The following metadata will show up in the PDF properties
\hypersetup{
  colorlinks = true,
  urlcolor = blue,
  pdfauthor = {\name},
  pdfkeywords = {computer science, curriculum vitae, \name, CV},
  pdftitle = {\name: Curriculum Vitae},
  pdfsubject = {Curriculum Vitae},
  pdfpagemode = UseNone
}

\geometry{
  a4paper,
  right=0.8in,
  left=0.5in,
  top=0.5in
}

\pagestyle{myheadings}
\markright{\name}
\thispagestyle{empty}

% Custom section fonts
\usepackage{sectsty}
\sectionfont{\rmfamily\mdseries\Large\scshape}
\subsectionfont{\rmfamily\mdseries\itshape\large\scshape}

% Don't indent paragraphs.
\setlength\parindent{0em}

% Make lists without bullets
\renewenvironment{itemize}{
  \begin{list}{}{
      \setlength{\leftmargin}{1em}
      \setlength{\itemsep}{4pt}
  }
}{
\end{list}
}

\newenvironment{positionsList}{
  \begin{itemize}
    \setlength{\itemsep}{20pt}
}{\end{itemize}}

% Listing environment with less spacing
\newenvironment{packed}{
\begin{itemize}
  \setlength{\itemsep}{0pt}
  \setlength{\parskip}{0pt}
  \setlength{\parsep}{0pt}
}{\end{itemize}}

% Increase `tabular` row spacing. Relevant for the tools/skills tables down below.
\renewcommand\arraystretch{1.4}

\begin{document}

\begin{minipage}{0.30\linewidth}
  {\huge \name}
  %\vspace{0.25in}
\end{minipage}
\begin{minipage}{0.75\linewidth}
  \begin{flushright}
    Email: \href{mailto:steffen@malignat.us}{\tt steffen@malignat.us} $\Vert$
    Portfolio: \href{https://github.com/MordecaiMalignatus/}{\tt
      github.com/MordecaiMalignatus}
  \end{flushright}
\end{minipage}

\section*{Experience}
\begin{positionsList}
\item {\large \textbf{Stripe}, Senior Software Engineer, {\footnotesize(Berlin, Germany)} \hfill \textbf{February 2024 {\textendash} August 2024}}
  \begin{itemize}
  \item Maintained core CI infrastructure under tight constraints and SLOs,
    built on top of AWS.\@
  \item Improved CI UX for Stripe's developers, reducing feedback cycle times
    and improving velocity.
  \item Migrated Docker across several major versions, updating a core component
    of the infrastructure with no downtimes and no notice to users.
  \item Interfaced with several teams to hunt down complex performance issues
    inside a distributed system.
  \end{itemize}
\item {\large \textbf{Terramate}, Senior Software Engineer, {\footnotesize(Berlin, Germany)} \hfill \textbf{June 2023 {\textendash} November 2023}}
  \begin{itemize}
  \item Developing and maintaining the Terramate CLI, importing ideas from build
    tools like Bazel and Nix to add features and improve performance.
  \item Creating and architecting the newly-created Terramate Cloud product
    backend, together with significant input on business and product
    development.
  \item Started and popularized incident reviews/retrospectives to learn from
    things having gone wrong.
  \end{itemize}
\item {\large \textbf{Wayfair}, Senior Platform Engineer, {\footnotesize(Berlin, Germany)} \hfill \textbf{July 2021 {\textendash} April 2023}}
  \begin{itemize}
  \item Maintaining and developing tools for Vault, Consul and Puppet at
    Wayfair as part of the Configuration Management Team.
  \item Developed a unified distribution platform on top of Vault, allowing
    other developers to not worry about secret distribution. For this we also
    built multiple interfaces for automated access and distribution.
  \item Planned, communicated and executed a several-major-versions upgrade of
    Vault with no downtime.
  \item Created and maintained developer-facing operational tools around Vault,
    incorporating principles of Cognitive Systems Engineering.
  \item Held multiple internal talks about the space of Resilience Engineering,
    Human Factors \& Ergonomics.
  \end{itemize}
\item {\large \textbf{PriceHubble}, Site Reliability Engineer, {\footnotesize(Berlin, Germany)}\hfill \textbf{November 2020 {\textendash} June 2021}}
  \begin{itemize}
    \item Maintained and developed tools for a large-scale Kubernetes
      cluster. This served as the basis of infrastructure to product teams.
    \item Started a cultural shift towards developer empowerment and autonomy,
      away from a classic Ops/Dev split. The incentives and final structure were
      modelled after the classic SRE literature.
    \item Transitioned company to new BI infrastructure based on GCP's
      BigQuery.
  \end{itemize}
\item {\large \textbf{Ryte}, Backend Software Engineer, {\footnotesize(Munich, Germany)}\hfill \textbf{May 2017 {\textendash} Sep 2019}}
   \begin{itemize}
      \item Maintained a Scala web crawler central to the company,
        distributed over YARN and Flink, analyzed by a Scala service
        written in Spark. This system is running on AWS EMR, and was
        subsequently improved to support Javascript rendering and crawling
        single-page applications by controlling Chromium.
  \end{itemize}
\end{positionsList}

\section*{Education}

\begin{itemize}
  \item \textbf{Technical University of Munich}, Incomplete B.Sc. Computer Science,  (2016
    {\textendash} 2018)
\end{itemize}

\section*{Technologies}
\begin{tabular}{ p{3cm} | p{13.75cm} }
  Programming \vfill Languages & Confident engineering skills in Go,
                                 Ruby, Rust, Python, and Elixir. \\
  Infrastructure \vfill as Code & As much as I can I rely on Infrastructure as Code, through
                                  whatever tool is most appropriate. These include: Terraform,
                                  Helm, CloudFormation, Puppet, Ansible, Nix. \\
  Kubernetes & I have maintained and developed for several clusters, mostly
               using GKE.\@Istio and Helm were also used.\\
  Platforms & I have used AWS and GCP to power large-scale CI systems, web
              crawlers, analysis pipelines and ML training clusters, website
              back-ends and have developed additional features for them
              based on the need of other team mates.\\
  Databases & Primarily worked with PostgreSQL and variations of SQL
              databases. In addition I also made use of BigQuery, DynamoDB,
              MongoDB, Kafka and various caches.\\
  CI/CD & Experience in setting up the software, process and tooling needed
          to continuously test and deploy services and products. Ability to
          troubleshoot flakey tests, reliability issues and aggressively taking
          aim at problems that make engineers less confident in CI.\\
\end{tabular}

\section*{Interests \& Focuses}
\begin{tabular}{ p{3cm} | p{13.75cm} }
  Observability & Modern, complex systems are hard to grasp and harder to change
                  successfully, and so having insight into what is happening in production is
                  invaluable. I build systems as transparent and observable as I
                  can. \\
  Cognitive Systems Engineering & I focus on the sociotechnical aspects of a
                                  system, rather than on either, human or
                                  computer, allowing me to optimize for the
                                  overall functionality. This means, for
                                  example, designing to avoid alert blindness,
                                  and designing to assist debugging in case of failure.\\
  Systems Design \& Resilience & I design systems for failure, which results in simple,
                                 but reliable architectures, easy to maintain and expand.\\
  Developer Tooling \& UX & I am very fond of having good tools to work with, and will create and
                            maintain tools that enable and simplify future
                            changes. \\
  Learning From \vfill Incidents & As frustrating as they are, production outages and
                                   incidents provide valuable opportunity for learning
                                   and improving, and as such, should not be wasted. I
                                   tend to set up incident reviews and analyses, and
                                   teach people how to do them. \\
\end{tabular}

\vfill

\begin{center}
  \begin{footnotesize}
    Last updated: \today
  \end{footnotesize}
\end{center}

\end{document}
