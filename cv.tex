\documentclass[a4paper]{article}

\usepackage{hyperref}
\usepackage[bottom=5em]{geometry}

%% TODO
% Datascience -> data science
% Split Technologies into Technologies and Interests
% List explicit tools and the problems they solved
% - nn
% - logrs
% Take out Gravitas until it has taken more shape
% fix in-word linebreaks


% Comment the following lines to use the default Computer Modern font
% instead of the Palatino font provided by the mathpazo package.
% Remove the 'osf' bit if you don't like the old style figures.
% \usepackage[T1]{fontenc}
% \usepackage[sc,osf]{mathpazo}

% Set your name here
\def\name{Steffen Reindl}

% Replace this with a link to your CV if you like, or set it empty
% (as in \def\footerlink{}) to remove the link in the footer:
\def\footerlink{http://git.malignat.us/Az/cv/raw/branch/master/cv.pdf}

% The following metadata will show up in the PDF properties
\hypersetup{
  colorlinks = true,
  urlcolor = blue,
  pdfauthor = {\name},
  pdfkeywords = {computer science, curriculum vitae, Steffen Reindl},
  pdftitle = {\name: Curriculum Vitae},
  pdfsubject = {Curriculum Vitae},
  pdfpagemode = UseNone
}

\geometry{
  body={6.5in, 8.5in},
  left=1.0in,
  top=1.25in
}

% Customize page headers
\pagestyle{myheadings}
\markright{\name}
\thispagestyle{empty}

% Custom section fonts
\usepackage{sectsty}
\sectionfont{\rmfamily\mdseries\Large\scshape}
\subsectionfont{\rmfamily\mdseries\itshape\large\scshape}

% Other possible font commands include:
% \ttfamily for teletype,
% \sffamily for sans serif,
% \bfseries for bold,
% \scshape for small caps,
% \normalsize, \large, \Large, \LARGE sizes.

% Don't indent paragraphs.
\setlength\parindent{0em}

% Make lists without bullets
\renewenvironment{itemize}{
  \begin{list}{}{
    \setlength{\leftmargin}{1.5em}
  }
}{
  \end{list}
}

% Listing environment with less spacing
\newenvironment{packed}{
\begin{itemize}
  \setlength{\itemsep}{0pt}
  \setlength{\parskip}{0pt}
  \setlength{\parsep}{0pt}
}{\end{itemize}}

\renewcommand\arraystretch{1.15}% (MyValue=1.0 is for standard spacing)

\begin{document}

{\huge \name}
\vspace{0.25in}

\begin{minipage}{0.45\linewidth}
  Naupliastr. 28\\
  81547, Munich\\
  Germany
\end{minipage}
\begin{minipage}{0.45\linewidth}
  \begin{tabular}{ll}
    Phone: & On request. \\
    Email: & \href{mailto:stc@malignat.us}{\tt stc@malignat.us} \\
    Portfolio: & \href{https://github.com/MordecaiMalignatus/}{\tt github.com/MordecaiMalignatus/}
  \end{tabular}
\end{minipage}

\section*{Skills and Experience}

\subsection*{Programming Languages}
\begin{tabular}{ l p{14cm} }
Scala & Go-to language for implementing services, specifically the Typelevel ecosystem,
also possessing experience of Akka-HTTP.\\
Python & Proficiency in building robust services and rapid prototyping,
experience in basic machine learning (Gensim/Keras) and datascience.\\
Elixir/OTP & Experience in building highly-available services to be used as
building blocks in an infrastructure, also some experience using Phoenix as
web-development backend.\\
Haskell & Experience in building highly-correct libraries and algorithm
implementations, and using it for prototyping and subsequent scaling.\\
Rust & Experience in creating CLI tools that do specific, small things well. \\
\end{tabular}

\subsection*{Technologies}
\begin{tabular}{ l p{11.7cm} }
 Functional Programming & I take cues and ideas from maths in order to reduce the
amount of bugs and complexity in my code.\\
Systems Design & I design systems I build for failure, which results in simple,
but reliable architectures, easy to maintain and expand.\\
Metrics & I like for services to have sensible and `actionable' metrics, for
example with DataDog \\
Infrastructure & I use tools like Terraform and CloudFormation to
construct all my infrastructure from day one. \\
AWS & I have used AWS in a professional setting to construct, load-balance and
maintain complex, distributed services. \\
Tooling & I am very fond of having good tools to work with, and will create and
maintain missing tools that simplify work to be done. \\
\end{tabular}

\section*{Employment}

\begin{itemize}
\item Backend Software Engineer at Ryte, 2017 {\textendash}
  present.
        \begin{itemize}
            \item Worked in an agile team developing Scala microservices.
            \item Conducted R\&D in a small team, developing algorithms,
              libraries and prototypes for the company in a variety of languages
              and technologies.
            \item Created and implemented information extraction algorithms,
              such as an evolution on PageRank.
            \item Used machine learning (Gensim/word2vec) to categorize and
              improve website content.
        \end{itemize}
\end{itemize}

\section*{Open-source work and self study}

\begin{itemize}
  \item Various small CLI tools oriented around managing documents, data and
    notes. (\href{https://github.com/MordecaiMalignatus?utf8=\%E2\%9C\%93\&tab=repositories\&q=\&type=public\&language=}{Github})
  \item Currently working through \href{http://haskellbook.com}{\emph{Haskell From First Principles}}
\item \href{https://github.com/LivewareProblems/gravitas}{Gravitas}
  \begin{packed}
    \item Cloud provisioning and compliance tool, for monitoring and controlling
      cloud changes.
    \item Positioned as alternative to Hashicorp's Terraform (eventually)
    \item Written in Elixir and Haskell
  \end{packed}
\end{itemize}

\section*{Education}

\begin{itemize}
  \item Incomplete B.Sc. Computer Science, Technical University of Munich (2016
    {\textendash} 2018)

  \item Abitur (A-levels), Asam Gymnasium, Munich
	\begin{packed}
	\item Average Grade: 2.8
	\item Excelled in computer science and English.
	\end{packed}
\end{itemize}


\section*{References}
Available on request.

\vfill

% Footer
\begin{center}
  \begin{footnotesize}
    Last updated: \today \\
    \href{\footerlink}{\texttt{\footerlink}}
  \end{footnotesize}
\end{center}

\end{document}
