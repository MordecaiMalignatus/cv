\documentclass[a4paper]{article}

\usepackage{hyperref}
\usepackage[bottom=5em]{geometry}

% \usepackage[T1]{fontenc}
% \usepackage[sc,osf]{mathpazo}

% Thomas' Suggestions:
% - Add "executive summary" to aid CTOs reviewing CVs, detailiing:
%   - How I got to where I am now
%   - What I do right now.
%   - What my personal interests are for programming.
%   - Rename Systems Design to Systems Architecture
% Inspiration: https://dianaolympos.github.io

\def\name{Steffen Reindl}

% Replace this with a link to your CV if you like, or set it empty
% (as in \def\footerlink{}) to remove the link in the footer:
\def\footerlink{http://git.malignat.us/Az/cv/raw/branch/master/cv.pdf}

% The following metadata will show up in the PDF properties
\hypersetup{
  colorlinks = true,
  urlcolor = blue,
  pdfauthor = {\name},
  pdfkeywords = {computer science, curriculum vitae, Steffen Reindl},
  pdftitle = {\name: Curriculum Vitae},
  pdfsubject = {Curriculum Vitae},
  pdfpagemode = UseNone
}

\geometry{
  a4paper,
  %% body={6.5in, 10in},
  right=0.75in,
  left=0.75in,
  top=0.75in
}

\pagestyle{myheadings}
\markright{\name}
\thispagestyle{empty}

% Custom section fonts
\usepackage{sectsty}
\sectionfont{\rmfamily\mdseries\Large\scshape}
\subsectionfont{\rmfamily\mdseries\itshape\large\scshape}

% Other possible font commands include:
% \ttfamily for teletype,
% \sffamily for sans serif,
% \bfseries for bold,
% \scshape for small caps,
% \normalsize, \large, \Large, \LARGE sizes.

% Don't indent paragraphs.
\setlength\parindent{0em}

% Make lists without bullets
\renewenvironment{itemize}{
  \begin{list}{}{
    \setlength{\leftmargin}{1.5em}
  }
}{
  \end{list}
}

% Listing environment with less spacing
\newenvironment{packed}{
\begin{itemize}
  \setlength{\itemsep}{0pt}
  \setlength{\parskip}{0pt}
  \setlength{\parsep}{0pt}
}{\end{itemize}}

\renewcommand\arraystretch{1.3}% (MyValue=1.0 is for standard spacing)

\begin{document}

{\huge \name}
\vspace{0.25in}

\begin{minipage}{0.45\linewidth}
  Naupliastr. 28\\
  81547, Munich\\
  Germany
\end{minipage}
\begin{minipage}{0.45\linewidth}
  \begin{tabular}{ll}
    Phone: & On request. \\
    Email: & \href{mailto:str@malignat.us}{\tt str@malignat.us} \\
    Portfolio: & \href{https://github.com/MordecaiMalignatus/}{\tt github.com/MordecaiMalignatus}
  \end{tabular}
\end{minipage}

\section*{Skills and Experience}

\subsection*{Programming Languages}
\begin{tabular}{ p{3cm} | p{13cm} }
Scala & Go-to language for implementing services, specifically the Typelevel ecosystem;
also possessing experience of Akka-HTTP.\\
Python & Proficiency in building robust services and rapid prototyping,
experience in basic machine learning (Gensim/Keras), and data science.\\
Elixir/OTP & Experience in building highly-available services to be used as
building blocks in an infrastructure; also some experience using Phoenix as
web-development backend.\\
Rust & Experience in creating CLI tools that do specific, small things well, as
well as agents/daemons for metrics. Also created a desktop application using
Rust and {\tt web-view} \\
\end{tabular}

\subsection*{Technologies}
\begin{tabular}{ p{3cm} | p{13cm} }
Metrics & What isn't measureable is not actionable. I add metrics and logs to
any service I write, as well as supporting infrastructure like dashboards and
log aggregation. \\
Infrastructure & I use tools like Terraform and CloudFormation to
construct all my infrastructure from day one. \\
AWS & I have used Amazon Web Services in a professional setting to construct, load-balance and
maintain complex, distributed services. \\
Databases & Primarily worked with PostgreSQL and variations of SQL
databases. I have also used NoSQL databases like Redis. \\
CI/CD & Experience in setting up the software, process and tooling needed
to continuously test and deploy services and products.\\
Docker & Experience packaging services to be used with Docker and
integrating Docker into the build process.
\end{tabular}

\subsection*{Interests \& Focuses}
\begin{tabular}{ p{3cm} | p{13cm} }
Functional \linebreak Programming & I take cues and ideas from maths in order to reduce the
amount of bugs and complexity in my code.\\
Systems Design & I design systems for failure, which results in simple,
but reliable architectures, easy to maintain and expand.\\
Tooling & I am very fond of having good tools to work with, and will create and
maintain tools that enable and simplify future changes. \\
\end{tabular}

\section*{Employment}

\begin{itemize}
\item \textbf{Ryte}, Backend Software Engineer, 2017 {\textendash}
  present.
        \begin{itemize}
            \item Conducted R\&D in a small team, developing algorithms,
              libraries and prototypes for the company in a variety of languages
              and technologies, where I used machine learning (Gensim/word2vec)
              to categorize and improve website content.
            \item Maintained a Scala web crawler central to the company,
              distributed over YARN and Flink, the outpout of which is analyzed
              by a Scala service written in Spark, running on AWS EMR. This
              crawler was subsequently improved to support Javascript rendering
              and crawling single-page applications.
        \end{itemize}
\end{itemize}

\pagebreak
\section*{Open-source work and self study}

\begin{itemize}
  \item logrs \href{https://github.com/MordecaiMalignatus/logrs}{(Github)}
    \begin{packed}
      \item Created to keep track of what was done during a day, for easy status
        reports and updates, as well as a way of recapturing of what was
        achieved. These notes are then grouped by day and archived in a central
        location.
    \end{packed}
  \item{Path of Beancounting \href{https://github.com/MordecaiMalignatus/path-of-beancounting}{(Github)}}
    \begin{packed}
    \item Light-weight desktop application for the game Path Of Exile, which
      gathers statistics from various sources, and displays them to the
      user. Most of it is written in Rust, while the front-end is written in
      {\tt web-view} with {\tt react.js}.
    \end{packed}
  \item Currently working through \href{http://haskellbook.com}{\emph{Haskell From First Principles}}
\end{itemize}

\section*{Education}

\begin{itemize}
  \item \textbf{Technical University of Munich}, Incomplete B.Sc. Computer Science,  (2016
    {\textendash} 2018)
  \item \textbf{Asam Gymnasium}, Abitur (A-levels), Munich
\end{itemize}

\section*{References}
Available on request.

\vfill

\begin{center}
  \begin{footnotesize}
    Last updated: \today \\
    \href{\footerlink}{\texttt{\footerlink}}
  \end{footnotesize}
\end{center}

\end{document}
